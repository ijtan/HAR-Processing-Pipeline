
\subsection{Sensing Devices}
When considering applications in Human Activity Recognition (HAR),  a range of sensors may be applied to recognise the activities being performed.  
One option is to use external sensors such as a camera, a microphone or as used in [A system for change detection and human
recognition in voxel space using the Microsoft Kinect sensor.] using the Microsoft Kinect as a 3D sensing device applied to Human Activity Recognition. 
However, these are usually limited to a static place, as the equipment to externally scan the activities needs a setup. On the other hand, 
there are wearable/portable sensing devices, such as accelerometers, gyroscopes or possibly heart rate monitors, which also have their own challenges. There were even methods such as 
[Emmanuel Tapia, Stephen Intille, Louis Lopez, and Kent Larson. The design of a portable kit of wireless sensors for naturalistic data collection. Pervasive Computing, pages 117-
134] where the two types of sensing devices were combined, aiming to analyze intricate activities.
\subsection{Data Acquisition}

During the process of data collection, care should be taken to keep conditions as natural as possible, as large deviation from 
natural conditions might make the model unusable, as conditions may be drastically different. The variety must also be maintained as to aim for a robust model 
which can handle different conditions, noise levels, users and environments 
[Human Activity and Motion Disorder Recognition:Towards Smarter Interactive Cognitive Environments Jorge L. Reyes-Ortiz1,2, Alessandro Ghio1, Davide Anguita1, Xavier Parra2, 
Joan Cabestany2, Andreu Catal`a2]. 


\subsection{Signal Processing \& Feature Extraction}
A robust signal processing pipeline is essential to reliably convert raw inertial data into a format usable by a classifier, as the features from the data need to be carefully extracted and highlighted to allow the classifiers to distinguish between the activities using the reproducible characteristics, such as statistical results on windows of data. 
Data windowing is a very common practice when considering sensory data, as several statistical tools are only unlocked on bundled data, and allow for richer and denser information per entry. An element of overlap between windows is also often employed, as the overlap allows for smoother transitions between activities. [Human Activity and Motion Disorder Recognition:Towards Smarter Interactive Cognitive Environments Jorge L. Reyes-Ortiz1,2, Alessandro Ghio1, Davide Anguita1, Xavier Parra2, Joan Cabestany2, Andreu Catal`a2].
Albeit the importance of reliable classification one also needs to consider the processing time required for the signal processing pipeline, as in some applications real-time classifications are necessary, and if the feature vector is excessively large, performance may be heavily impacted. 



% \pagebreak
\subsection{Support Vector Machines}
    Support Vector Machines (SVM) are a type of classifier used for classification and regression. SVM seeks to classify samples as different classes
    on a hyperplane in multidimensional space. The classification process generates multiple hyperplanes to find the one which minimises an error value.
    Generally, support vector machines are used for binary classification problems. However, they can be adapted to multiclass problems by converting
    or reducing them into a set of multiple binary classification problems. However some samples cannot be classified linearly, hence SVMs can also make
    use of kernels to transform non-linear space into linear space, which was proposed by Boser et al. \cite{Boser1992}.


\subsection{Convolutional Neural Networks}
The Convolutional Neural Network (CNN) is a type of deep learning neural network model commonly applied for computer vision applications.
Like other deep learning neural networks, CNNs consist of an input layer, any number of hidden layers, and an output layer.
Unique to the CNN is the use of the convolutional layer, these layers use a kernel that goes over the input tensor, causing the convolution.
Fully connected linear layers can achieve the same results however, for large inputs FC layers will have many weights.
With convolutional layers the number of these weights is reduced.
Each layer has its own activation function, which acts as a filter for whether a neuron should fire or not.
In this project the RELU, Softmax and LogSoftmax activations where used.
After a number of convolutional layers an FC layer is usually inserted, and in this project each CNN has an FC layer at the end.
