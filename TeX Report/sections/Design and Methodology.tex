\subsection{Feature Extraction}
    Excluding the label for each activity, each session in the dataset contains 589 features. These features were extracted by applying a number of different statistical
    measures to the different extracted signals in each sliding window.

    Each signal has the statistical measures in Table 1 applied to them.

    \begin{table}[ht]
        \begin{tabular}{|l|l|lll}
            \cline{1-2}
            \textbf{Statistical Measure} & \textbf{Description} &  &  &  \\ \cline{1-2}
            mean             & Mean value in the window           &  &  &  \\ \cline{1-2}
            min            & Smallest value in the window           &  &  &  \\ \cline{1-2}
            max            & Largest value in the window           &  &  & \\ \cline{1-2}
            std            & Standard deviation           &  &  & \\ \cline{1-2}
            entropy            & Entropy           &  &  & \\ \cline{1-2}
            mad            & 3           &  &  & \\ \cline{1-2}
            iqr            & Inter-quartile Range           &  &  & \\ \cline{1-2}
            energy            & 3           &  &  & \\ \cline{1-2}
            sma            & Signal magnitude area           &  &  & \\ \cline{1-2}
            arCoeff            & 3           &  &  & \\ \cline{1-2}
            correlation            & 3           &  &  & \\ \cline{1-2}
            angle            & Angle between 2 vectors of signals           &  &  & \\ \cline{1-2}
            band energy            & 3           &  &  & \\ \cline{1-2}
        \end{tabular}
        \caption*{Table 1}
    \end{table}

    The statistical measures in Table 2 are applied only to FFT signals.

    \begin{table}[ht]
        \begin{tabular}{|l|l|lll}
            \cline{1-2}
            \textbf{Statistical Measure} & \textbf{Description} &  &  &  \\ \cline{1-2}
            maxInds             & 1           &  &  &  \\ \cline{1-2}
            skewness            & 2           &  &  &  \\ \cline{1-2}
            kurtosis               & 3           &  &  &  \\ \cline{1-2}
            meanFreq            & 3           &  &  &  \\ \cline{1-2}
        \end{tabular}
        \caption*{Table 2}
    \end{table}


\subsection{Signal Processing}

\subsection{Support Vector Classifier}
    \subsubsection{Data Pre-processing}
        Firstly, the training and testing datasets were converted to Pandas dataframes and the labels for each activity were separated into their own variables.
        Using a LabelEncoder, each activity label is converted into a numerical value.

    \subsubsection{Comparing different classification models}
        In total, four different classic machine learning classifiers were used on the dataset developed by Anguita et al \cite{Anguita2012}. These classifiers are Gaussian Naïve Bayes,
        AdaBoost, Stochastic Gradient Descent and a Support Vector Classifier. These were trained and tested using the respective datasets, and their accuracy,
        F-beta, precision and recall scores were recorded.

        \begin{table}[ht]
            \centering\small
            \begin{tabular}{|l|l|l|l|l|}
                \hline
                \textbf{Classifier} & \textbf{Accuracy} & \textbf{F-Beta}  & \textbf{Precision} & \textbf{Recall} \\ \hline
                Gaussian Naïve Bayes             & 0.7134           & 0.7252  & 0.7555 & 0.7134 \\ \hline
                AdaBoost            & 0.4065           & 0.2520 & 0.4289 & 0.4065 \\ \hline
                Stochastic Gradient Descent               & 0.9600           & 0.9603 & 0.9605 & 0.9600 \\ \hline
                Support Vector Classifier            & 0.9668           & 0.9676 & 0.9682 & 0.9668 \\ \hline
            \end{tabular}
            \caption*{Table 3: Performance scores of each classifier}
        \end{table}

        As can be seen in Table 3, the Support Vector Classifier had the highest scores, each result being over 0.96. It is for this reason that the SVC
        was chosen to classify the UCI dataset \cite{Anguita2013} and, later on, the dataset built by ourselves.

    \subsection{Classification}
        The RBF kernel is defined as the exponential function \(exp(-\gamma \lvert x-x' \rvert)^2\) A primer on kernel methods, JP Vert et al, where x and x’ are two feature vectors, and is the
        gamma parameter in the classifier. Gamma’s value is scale, meaning that the parameter is the reciprocal of the number of features multiplied with the variance of the input data.
        For this implementation, we opted for a One-Vs-All approach. A One-Vs-All approach divides the data points into just two classes: a certain activity X and the other classes. Therefore records 
        labelled as \emph{SITTING} are a single class, and the other activities are treated as having a single label.
