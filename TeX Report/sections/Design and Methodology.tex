\subsection{Feature Extraction}
    Excluding the label for each activity, each session in the dataset contains 589 features. These features were extracted by applying a number of different statistical
    measures to the different extracted signals in each sliding window.

    Each signal has the following statistical measures applied to them.

    \begin{table}[]
    \begin{tabular}{|l|l|lll}
    \cline{1-2}
    Statistical Measure & Description &  &  &  \\ \cline{1-2}
    mean             & Mean value in the window           &  &  &  \\ \cline{1-2}
    min            & Smallest value in the window           &  &  &  \\ \cline{1-2}
    max            & Largest value in the window           &  &  & \\ \cline{1-2}
    std            & Standard deviation           &  &  & \\ \cline{1-2}
    entropy            & Entropy           &  &  & \\ \cline{1-2}
    mad            & 3           &  &  & \\ \cline{1-2}
    iqr            & Inter-quartile Range           &  &  & \\ \cline{1-2}
    energy            & 3           &  &  & \\ \cline{1-2}
    sma            & Signal magnitude area           &  &  & \\ \cline{1-2}
    arCoeff            & 3           &  &  & \\ \cline{1-2}
    correlation            & 3           &  &  & \\ \cline{1-2}
    angle            & Angle between 2 vectors of signals           &  &  & \\ \cline{1-2}
    band energy            & 3           &  &  & \\ \cline{1-2}
    \end{tabular}
    \caption{Table 1}
    \label{tab:table-1}
    \end{table}

    The following statistical measures are applied only to FFT signals

    \begin{table}[]
    \begin{tabular}{|l|l|lll}
    \cline{1-2}
    Statistical Measure & Description &  &  &  \\ \cline{1-2}
    maxInds             & 1           &  &  &  \\ \cline{1-2}
    skewness            & 2           &  &  &  \\ \cline{1-2}
    kurtosis               & 3           &  &  &  \\ \cline{1-2}
    meanFreq            & 3           &  &  &  \\ \cline{1-2}
    \end{tabular}
    \caption{Table 2}
    \label{tab:table-2}
    \end{table}


\subsection{Signal Processing}

\subsection{Support Vector Classifier}
    \subsubsection{Data Pre-processing}
        

    \subsubsection{Comparing different classification models}
        In total, four different classic machine learning classifiers were used on the dataset developed by Anguita et al \cite{Anguita2012}. These classifiers are Gaussian Naïve Bayes,
        AdaBoost, Stochastic Gradient Descent and a Support Vector Classifier. These were trained and tested using the respective datasets, and their accuracy,
        F-beta, precision and recall scores were recorded.

    \subsection{Classification}