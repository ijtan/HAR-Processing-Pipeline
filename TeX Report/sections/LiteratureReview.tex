Anguit et al. \cite{Anguita2012}  classified activities into two different categories:  static and dynamic activities.
Using a One-Vs-All approach and a Laplacian kernel, they obtained highly accurate rates of classification.
Demrozi et al. \cite{Demrozi2020} compiled a number of studies on human activity recognition (Human Activity Recognition using Inertial, Physiological and Environmental Sensors: A Comprehensive Survey) and found that out of 149 papers published between January 2015 and September 2019, 53 examined deep learning models and the remaining 96 were classical machine learning models.
Cho et al. \cite{Cho2018} propose a divide and conquer approach using a 1-dimensional convolutional neural network on the UCI dataset.
In \cite{Anguita2013EnergyES}, floating-point arithmetic was exploited on smartphones to achieve a considerable increase in battery life whilst maintaining similar accuracy to traditional methods of human activity classification.
Whereas \cite{ReyesOrtiz2013HumanAA} porposed the exploitation of Human activity recognition techniques to detect motion disorders such as Parkinson's disease through sensory data.
